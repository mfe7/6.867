\documentclass[letterpaper, 10 pt, conference]{ieeeconf}
\IEEEoverridecommandlockouts % This command is only needed if
% you want to use the \thanks command

\overrideIEEEmargins % Needed to meet printer requirements.

% See the \addtolength command later in the file to balance the column
% lengths on the last page of the document
\usepackage{microtype}
% The following packages can be found on http:\\www.ctan.org
% \usepackage{graphics} % for pdf, bitmapped graphics files
% \usepackage{epsfig} % for postscript graphics files
% \usepackage{mathptmx} % assumes new font selection scheme installed
% \usepackage{times} % assumes new font selection scheme installed
% \usepackage{amsmath} % assumes amsmath package installed
% \usepackage{amssymb} % assumes amsmath package installed

\usepackage[pdftex,pdfauthor={Michael Everett},pdftitle={6.867 Pset 1}]{hyperref}
\hypersetup{colorlinks,linkcolor={green!50!black},citecolor={green!50!black},urlcolor={blue!80!black}}
\makeatletter \let\NAT@parse\undefined \makeatother
% \usepackage[square,comma,sort&compress]{natbib}
\usepackage[sort,compress]{cite}
\usepackage{graphicx} % more modern
\usepackage{amsfonts}
\usepackage{amsmath,soul}
\usepackage{color}
\usepackage[font=small]{subcaption}
\usepackage{balance}
\usepackage[font=small]{caption}
%\usepackage{subfigure,balance}
%\usepackage[colorlinks=true]{hyperref}
%\usepackage{subcaption,balance}
%\usepackage{algorithm} \usepackage[noend]{algorithmic}
\usepackage[linesnumbered,ruled,vlined]{algorithm2e}
\usepackage{multirow}

\DeclareMathOperator*{\argmin}{\arg\!\min}
\DeclareMathOperator*{\argmax}{\arg\!\max}
\usepackage{tabulary}
\newcolumntype{K}[1]{>{\centering\arraybackslash}p{#1}}


\newtheorem{definition}{Definition}
\newtheorem{assumption}{Assumption} \newtheorem{theorem}{Theorem}
\newtheorem{lemma}{Lemma}
\newtheorem{corollary}[theorem]{Corollary}

\title{\LARGE \bf 6.867 : Homework 1}

\author{Anonymous authors}

% \usepackage[usenames]{color}
%\DeclareMathOperator*{\argmin}{arg\,\!min}
%\DeclareMathOperator*{\argmax}{arg\,\!max}
\usepackage[svgnames]{xcolor} \definecolor{DarkGreen}{rgb}{0,0.5,0}
\definecolor{DarkRed}{rgb}{0.75,0,0}

\usepackage[authormarkuptext=name,addedmarkup=bf,authormarkupposition=left]{changes}
%\usepackage[final]{changes} %Use this to hide all comments.
\definechangesauthor[name={M.~E.}, color={red}]{me}
\setremarkmarkup{(#2)}

%\newcommand{\mXX}[1]{{\color{DarkRed} \bf XX #1 XX\ }}
\newcommand{\mXX}[1]{\added[id=ml,remark={}]{#1}}
\newcommand{\sXX}[1]{\added[id=sc,remark={}]{#1}}
%\newcommand{\XX}[1]{{\bf \color{red} XX #1 XX}}
\newcommand{\XX}[1]{\added[id=jh,remark={}]{#1}}
\newcommand{\jmXX}[1]{\added[id=jm,remark={}]{#1}}
\newcommand{\meXX}[1]{\added[id=me,remark={}]{#1}}



\newcommand{\jsec}[1]{\marginpar{\fcolorbox{yellow}{yellow}{\parbox{0.7in}{\raggedright
        \color{blue} \tiny #1 }}}}
\newcommand{\hsec}[1]{\marginpar{\fcolorbox{yellow}{yellow}{\parbox{0.7in}{\raggedright
        \color{green} \tiny #1 }}}}
\newcommand{\jhmargin}[2]{{\color{orange}#1}\marginpar{\color{orange}\tiny\raggedright
    \bf [JH] #2}}


\usepackage{tikz,mathtools}
%\usepackage{cleveref}
\usepackage[capitalize]{cleveref}
\crefformat{equation}{(#2#1#3)}
\Crefformat{equation}{Equation~(#2#1#3)}
\Crefname{equation}{Equation}{Equations}

\newcommand{\inputTikZ}[2]{\scalebox{#1}{\input{#2}}}
\usetikzlibrary{shapes,positioning,automata,arrows,fit,backgrounds,calc}
\tikzstyle{block} = [draw, fill=blue!20, rectangle,minimum height=1em,
minimum width=2em] \tikzstyle{sum} = [draw, fill=blue!20, circle, node
distance=1cm] \tikzstyle{input} = [coordinate] \tikzstyle{output} =
[coordinate] \tikzstyle{pinstyle} = [pin edge={to-,thin,black}]
\usetikzlibrary{trees} \usetikzlibrary{decorations.pathmorphing}
\usetikzlibrary{decorations.markings}
\definecolor{darkgreen}{rgb}{0,0.5,0}
\definecolor{darkred}{rgb}{220,20,60}

\makeatletter
\renewcommand\paragraph{\@startsection{subsubsection}{4}{\z@}%
{0.25ex \@plus.5ex \@minus.2ex}%
{-.15em}%
{\normalfont\normalsize\itshape}}
\makeatother


\begin{document}

\maketitle
\thispagestyle{empty} \pagestyle{empty}

% %%%%%%%%%%%%%%%%%%%%%%%%%%%%%%%%%%%%%%%%%%%%%%%%%%%%%%%%%%%%%%%%%%%%%%%%%%%%%%%
% \begin{abstract} 
% Abstract here.
% \end{abstract}

%%%%%%%%%%%%%%%%%%%%%%%%%%%%%%%%%%%%%%%%%%%%%%%%%%%%%%%%%%%%%%%%%%%%%%%%%%%%%%%%
%!TEX root = main.tex
\section{Neural Networks} \label{sec:prob1}
In this problem, we implement a simple neural network in Python.
The network is trained with backpropogation and stochastic gradient descent.

\subsection{Part 1}
In this problem, we are mainly interested in classification tasks.
Thus, we pass the output of the final hidden layer through a softmax layer to generate a probability (outputs are positive and sum to 1) vector where each element $k$ is the probability of the input being in class $k$.

In general, the training objective is to minimize loss, so at each training step, we evaluate the loss function and compute its gradient, and then adjust the weights and biases in a direction to decrease loss according to the learning rate.
According to the backpropogation algorithm, the derivative of loss with respect to final activation is $\delta^L = Diag[f'(z)] \nabla_a loss$.
For this network with softmax output activation and cross-entropy loss function, we can compute $\delta^L = p(x) - y$, where p(x) is the predicted output for a particular training data point, $x$ in class $y$ (where p is also a function of the network weights, and other parameters).


\subsection{Part 2}
According to the Xavier Initialization, we can initialize weights to be a zero-mean Gaussian with variance [todo].
We initialize biases to zero, but if we initialize weights to zero as well, [todo].

\subsection{Part 3}
To add weight regularization, the objective function would look like:
\begin{equation}
J(w) = l(w) + \lambda(||w^{(1)}||^2_F + ||w^{(2)}||^2_F)
\end{equation}
The only change to the pseudo-code from the lecture notes would be an updated gradient term.
[todo]: gradient term.






% \input{background}
% \input{approach}
% \input{results}
% \input{conclusion}
%\listofchanges

%\balance

% \addtolength{\textheight}{-12cm} % This command serves to balance the column lengths
% on the last page of the document manually. It shortens
% the textheight of the last page by a suitable amount.
% This command does not take effect until the next page
% so it should come on the page before the last. Make
% sure that you do not shorten the textheight too much.

%%%%%%%%%%%%%%%%%%%%%%%%%%%%%%%%%%%%%%%%%%%%%%%%%%%%%%%%%%%%%%%%%%%%%%%%%%%%%%%%



%%%%%%%%%%%%%%%%%%%%%%%%%%%%%%%%%%%%%%%%%%%%%%%%%%%%%%%%%%%%%%%%%%%%%%%%%%%%%%%%



%%%%%%%%%%%%%%%%%%%%%%%%%%%%%%%%%%%%%%%%%%%%%%%%%%%%%%%%%%%%%%%%%%%%%%%%%%%%%%%%
% \section*{Acknowledgment}
% This work is supported by Ford Motor Company.
%%%%%%%%%%%%%%%%%%%%%%%%%%%%%%%%%%%%%%%%%%%%%%%%%%%%%%%%%%%%%%%%%%%%%%%%%%%%%%%%
%\clearpage
\balance
% \bibliographystyle{IEEEtran} 
% %\bibliographystyle{unsrt} 
% \bibliography{biblio}
% %\balance
\end{document}
%\grid
