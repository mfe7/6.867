\documentclass[letterpaper, 10 pt, conference]{ieeeconf}
\IEEEoverridecommandlockouts % This command is only needed if
% you want to use the \thanks command

\overrideIEEEmargins % Needed to meet printer requirements.

% See the \addtolength command later in the file to balance the column
% lengths on the last page of the document
\usepackage{microtype}
% The following packages can be found on http:\\www.ctan.org
% \usepackage{graphics} % for pdf, bitmapped graphics files
% \usepackage{epsfig} % for postscript graphics files
% \usepackage{mathptmx} % assumes new font selection scheme installed
% \usepackage{times} % assumes new font selection scheme installed
% \usepackage{amsmath} % assumes amsmath package installed
% \usepackage{amssymb} % assumes amsmath package installed

\usepackage[pdftex,pdfauthor={anonymous},pdftitle={6.867 Pset 1}]{hyperref}
\hypersetup{colorlinks,linkcolor={green!50!black},citecolor={green!50!black},urlcolor={blue!80!black}}
\makeatletter \let\NAT@parse\undefined \makeatother
% \usepackage[square,comma,sort&compress]{natbib}
\usepackage[sort,compress]{cite}
\usepackage{graphicx} % more modern
\usepackage{amsfonts}
\usepackage{amsmath,soul}
\usepackage{color}
\usepackage[font=small]{subcaption}
\usepackage{balance}
\usepackage[font=small]{caption}
%\usepackage{subfigure,balance}
%\usepackage[colorlinks=true]{hyperref}
%\usepackage{subcaption,balance}
%\usepackage{algorithm} \usepackage[noend]{algorithmic}
\usepackage[linesnumbered,ruled,vlined]{algorithm2e}
\usepackage{multirow}

% from bjorn:
\usepackage{xfrac}
\usepackage{mathtools}
\usepackage{bm}


\DeclareMathOperator*{\argmin}{\arg\!\min}
\DeclareMathOperator*{\argmax}{\arg\!\max}
\usepackage{tabulary}
\newcolumntype{K}[1]{>{\centering\arraybackslash}p{#1}}


\newtheorem{definition}{Definition}
\newtheorem{assumption}{Assumption} \newtheorem{theorem}{Theorem}
\newtheorem{lemma}{Lemma}
\newtheorem{corollary}[theorem]{Corollary}

\title{\LARGE \bf 6.867: Homework 3}

\author{Anonymous authors}

% \usepackage[usenames]{color}
%\DeclareMathOperator*{\argmin}{arg\,\!min}
%\DeclareMathOperator*{\argmax}{arg\,\!max}
\usepackage[svgnames]{xcolor} \definecolor{DarkGreen}{rgb}{0,0.5,0}
\definecolor{DarkRed}{rgb}{0.75,0,0}

\usepackage[authormarkuptext=name,addedmarkup=bf,authormarkupposition=left]{changes}
%\usepackage[final]{changes} %Use this to hide all comments.
\definechangesauthor[name={M.~E.}, color={red}]{me}
\setremarkmarkup{(#2)}

%\newcommand{\mXX}[1]{{\color{DarkRed} \bf XX #1 XX\ }}
\newcommand{\mXX}[1]{\added[id=ml,remark={}]{#1}}
\newcommand{\sXX}[1]{\added[id=sc,remark={}]{#1}}
%\newcommand{\XX}[1]{{\bf \color{red} XX #1 XX}}
\newcommand{\XX}[1]{\added[id=jh,remark={}]{#1}}
\newcommand{\jmXX}[1]{\added[id=jm,remark={}]{#1}}
\newcommand{\meXX}[1]{\added[id=me,remark={}]{#1}}



\newcommand{\jsec}[1]{\marginpar{\fcolorbox{yellow}{yellow}{\parbox{0.7in}{\raggedright
        \color{blue} \tiny #1 }}}}
\newcommand{\hsec}[1]{\marginpar{\fcolorbox{yellow}{yellow}{\parbox{0.7in}{\raggedright
        \color{green} \tiny #1 }}}}
\newcommand{\jhmargin}[2]{{\color{orange}#1}\marginpar{\color{orange}\tiny\raggedright
    \bf [JH] #2}}


\usepackage{tikz,mathtools}
%\usepackage{cleveref}
\usepackage[capitalize]{cleveref}
\crefformat{equation}{(#2#1#3)}
\Crefformat{equation}{Equation~(#2#1#3)}
\Crefname{equation}{Equation}{Equations}

\newcommand{\inputTikZ}[2]{\scalebox{#1}{\input{#2}}}
\usetikzlibrary{shapes,positioning,automata,arrows,fit,backgrounds,calc}
\tikzstyle{block} = [draw, fill=blue!20, rectangle,minimum height=1em,
minimum width=2em] \tikzstyle{sum} = [draw, fill=blue!20, circle, node
distance=1cm] \tikzstyle{input} = [coordinate] \tikzstyle{output} =
[coordinate] \tikzstyle{pinstyle} = [pin edge={to-,thin,black}]
\usetikzlibrary{trees} \usetikzlibrary{decorations.pathmorphing}
\usetikzlibrary{decorations.markings}
\definecolor{darkgreen}{rgb}{0,0.5,0}
\definecolor{darkred}{rgb}{220,20,60}

\makeatletter
\renewcommand\paragraph{\@startsection{subsubsection}{4}{\z@}%
{0.25ex \@plus.5ex \@minus.2ex}%
{-.15em}%
{\normalfont\normalsize\itshape}}
\makeatother


\begin{document}

\maketitle
\thispagestyle{empty} \pagestyle{empty}

% %%%%%%%%%%%%%%%%%%%%%%%%%%%%%%%%%%%%%%%%%%%%%%%%%%%%%%%%%%%%%%%%%%%%%%%%%%%%%%%

\begin{abstract}
This is Machine Learning homework 3. Topics include neural networks, conv nets, and LDA. All work has been done in Python.
\end{abstract}

%%%%%%%%%%%%%%%%%%%%%%%%%%%%%%%%%%%%%%%%%%%%%%%%%%%%%%%%%%%%%%%%%%%%%%%%%%%%%%%%
% \section{Gradient Descent} \label{sec:prob1}
In this section, we implemented Gradient Descent in Python to find the minimum of a function.

\subsection{Part 1}
Our Gradient Descent (GD) implementation begins with an initial guess, $x_0$, of the argument $x_{min}$ that minimizes the function, $f(x)$.
The implementation is applied to two functions, a negative Gaussian and a quadratic bowl.
The gradient, dfdx, can be calculated in closed form (for simple functions), and the next estimate of $x_{min}$ is updated according to the rule $x_{i+1} = x_i - \alpha \cdot \frac{\partial f(x_i)}{\partial x}$.
Here, $\alpha$ is the step size, or learning rate, which controls how much the gradient affects the next estimate.
The rule is applied until either a pre-determined maximum number of iterations, $n_{iters}$, occurs, or the function's value at the current iteration differs from the previous iteration's estimate by less than $\epsilon$.
Each of these paramters, ($\alpha$, $n_{iters}$, and $\epsilon$) impact the GD solution in different ways.

In~\cref{fig:initial_guess}, the intial guess, $x_0$, is varied to show that a close intial guess leads to convergence in only a few steps.
As the norm of difference between initial and final estimates grows, the number of iterations increases.
For certain initial guesses, we observed that the algorithm did not converge in some maximum number of iterations.
For the negative Gaussian, this is because the gradient is very small far from the mean.

The convergence limit determines when the GD algorithm stops, seen in ~\cref{fig:convergence}.
The error increases if $\epsilon$ is too large, as on the right of~\cref{fig:convergence1}.

In GD, the norm of the gradient decreases as the solution converges, which is shown for the quadratic bowl function in~\cref{fig:gradient}.


\begin{figure}
	\centering
	\includegraphics [trim=0 0 0 0, clip, angle=0, width=0.8\columnwidth,
	keepaspectratio]{figures/1_1_convergence1}
	\caption{As convergence limit, $\epsilon$, increases, the iteration stops earlier. From this view, the four solutions differ very little.} 
	\label{fig:convergence1} 
	%\vskip -0.1in
\end{figure}

\begin{figure}
	\centering
	\includegraphics [trim=0 0 0 0, clip, angle=0, width=0.8\columnwidth,
	keepaspectratio]{figures/1_1_convergence}
	\caption{The final value returned by GD can differ substantially if convergence limit is set too loosely.} 
	\label{fig:convergence} 
	%\vskip -0.1in
\end{figure}

\begin{figure}
	\centering
	\includegraphics [trim=0 0 0 0, clip, angle=0, width=0.8\columnwidth,
	keepaspectratio]{figures/1_1_initial_guess}
	\caption{When initial guess is far from true minimum, it takes many iterations (blue) to converge. As initial guess approaches final solution, number of iterations decreases.} 
	\label{fig:initial_guess} 
	%\vskip -0.1in
\end{figure}

\begin{figure}
	\centering
	\includegraphics [trim=0 0 0 0, clip, angle=0, width=0.8\columnwidth,
	keepaspectratio]{figures/1_1_step_size}
	\caption{If learning rate, $\alpha$, is too large (blue), the solution will oscillate or diverge each iteration. If $\alpha$ is too small, it will take many iterations to converge (yellow). Convergence takes only a few iterations for proper $\alpha$ (red).} 
	\label{fig:learning_rate} 
	%\vskip -0.1in
\end{figure}

\begin{figure}
	\centering
	\includegraphics [trim=0 0 0 0, clip, angle=0, width=0.8\columnwidth,
	keepaspectratio]{figures/1_1_gradient}
	\caption{Norm of gradient decreases with each iteration in GD. This example minimizes the Quadratic Bowl function.} 
	\label{fig:gradient} 
	%\vskip -0.1in
\end{figure}












\section{Convolutional Neural Network (CNN)} \label{sec:prob2}
In this section, we consider the Convolutional Neural Network (CNN) to perform artist identification on paintings.

\subsection{Part 1}
In a convolutional filter, if the first layer applies a 5x5 patch to the image to generate feature $Z_1$, and the second layer applies a 3x3 patch to feature $Z_1$ to generate feature $Z_2$, the receptive field of $Z_2$ (or dimensions of image that affect the node) is 7x7.
That is, a window of 49 neighboring pixels in the original image affects a single node at the output of the filter.
This allows the network to learn spatial features from the original image.
If the conv net becomes deeper (more layers), the network can use larger and more complex combinations of features/regions of the image.

\subsection{Part 2}
We are provided with a conv net (conv.py).
In total, there are [todo] layers: 2 convolutional layers, 1 flatten layer, and 2 dense layers.
The output is the maximum logit of the final dense layer.
[todo] confirm activation function. The hidden layer activation function is relu.
The loss function is softmax cross entropy with logits.
Loss is minimized with Gradient Descent (with a tunable learning rate parameter).

The provided network took about 45 seconds to train on a Macbook CPU.
After 1500 steps, the training accuracy is 87.4\%, and the validation accuracy is 57.5\%.
These numbers suggest overfitting, because the model does not generalize to unseen data very well.

\subsection{Part 3}
\section{Pegasos} \label{sec:prob3}
In this section, we use the Pegasos algorithm for soft-margin SVM binary classification of one of the 2D datasets.

\subsection{Part 1}
The Pegasos algorithm is a method to solve soft-margin SVM.
We implemented the algorithm according to the provided pseudocode, and added a bias term, $w_0$ that is incremented by $\eta_ty_i$ for misclassified sample $x_i$ each iteration.
The bias term is not penalized by the regularization.
The formulation solved by Pegasos is:
\begin{equation}\label{eq:svm_pegasos}
min_w \frac{\lambda}{2}||w||^2 + \frac{1}{N}\sum\limits_{i=1}^{N}max\{0,1-y_i(w^Tx_i)\}
\end{equation}

\subsection{Part 2}
The algorithm is applied to the same 2D dataset for several values of $\lambda$ ($\lambda = [2, 2^{-1}, 2^{-2}, 2^{-4}$), shown in~\cref{fig:3_2_lambdas}.
For large $\lambda$ (left), magnitude of $w$ is penalized heavily, so the decision boundary is not very accurate.
As $\lambda$ decreases, the regularization penalty becomes less dominant compared with accuracy, so the accuracy improves.
This intuition agrees with the objective function~\cref{eq:svm_pegasos}.

\begin{figure}
	\centering
	\includegraphics [trim=0 0 0 0, clip, angle=0, width=0.8\columnwidth,
	keepaspectratio]{figures/3_2_lambdas}
	\caption{The Pegasos algorithm is applied to the same dataset with four different regularization parameter values, $\lambda$. For large $\lambda$ (left), magnitude of $w$ is penalized heavily, so the decision boundary is not very accurate. As $\lambda$ decreases, the regularization penalty becomes less dominant compared with accuracy, so the accuracy improves.}
	\label{fig:3_2_lambdas} 
\end{figure}

\subsection{Part 3}
Next, we extended our implementation to handle a kernel matrix input, and tested it with a gaussian RBF kernel as in~\cref{sec:prob2}.
After learning $\alpha$, we predict the class of a new sample, $x_i$ by first calculating $c = \alpha \cdot K(X, x_i; \gamma)$, where $X$ is the entire training set and $\gamma$ is the Gaussian kernel bandwidth.
If $c > 0$, it's part of the positive class, otherwise it's the negative class.

\subsection{Part 4}
We tested our algorithm on multiple values of $\gamma$ with a fixed $\lambda=0.02$.
For large $\gamma$ (left), the kernel is very narrow, and the decision boundary overfits the data, evident in the decision boundary's jagged shape tracing around individual borderline samples.
As $\gamma$ decreases, the accuracy decreases, but the decision boundary is much less complicated.
This usually leads to better generalization to unseen data.

The number of support vectors decreases as $\gamma$ increases (192, 165, 10, 22).
This observation aligns with the decreasing complexity of the decision boundary, because each support vector affects the shape of the decision boundary.

These observations associated with increasing $\gamma$ match the trends seen in~\cref{sec:prob2} with increasing $C$.
Both implementations (Pegasos and SVM) are able to correctly classify the dataset.
Our Pegasos implementation runs much faster, making it easier to test with different parameters for increased performance.


\begin{figure}
	\centering
	\includegraphics [trim=0 0 0 0, clip, angle=0, width=0.8\columnwidth,
	keepaspectratio]{figures/3_3_decisions}
	\caption{The Pegasos algorithm is applied to the same dataset with four different kernel bandwidth values, $\gamma = [2^2, 2^1, 2^0, 2^{-1}]$. For large $\gamma$ (left), the kernel is narrow, and the decision boundary overfits the data. As $\gamma$ decreases, the accuracy decreases, but the decision boundary is much less complicated.}
	\label{fig:3_3_decisions} 
\end{figure}
%\listofchanges

%\balance

% \addtolength{\textheight}{-12cm} % This command serves to balance the column lengths
% on the last page of the document manually. It shortens
% the textheight of the last page by a suitable amount.
% This command does not take effect until the next page
% so it should come on the page before the last. Make
% sure that you do not shorten the textheight too much.

%%%%%%%%%%%%%%%%%%%%%%%%%%%%%%%%%%%%%%%%%%%%%%%%%%%%%%%%%%%%%%%%%%%%%%%%%%%%%%%%



%%%%%%%%%%%%%%%%%%%%%%%%%%%%%%%%%%%%%%%%%%%%%%%%%%%%%%%%%%%%%%%%%%%%%%%%%%%%%%%%



%%%%%%%%%%%%%%%%%%%%%%%%%%%%%%%%%%%%%%%%%%%%%%%%%%%%%%%%%%%%%%%%%%%%%%%%%%%%%%%%
% \section*{Acknowledgment}
% This work is supported by Ford Motor Company.
%%%%%%%%%%%%%%%%%%%%%%%%%%%%%%%%%%%%%%%%%%%%%%%%%%%%%%%%%%%%%%%%%%%%%%%%%%%%%%%%
%\clearpage
\balance
% \bibliographystyle{IEEEtran} 
% %\bibliographystyle{unsrt} 
% \bibliography{biblio}
% %\balance
\end{document}
%\grid
