%!TEX root = main.tex

\section{Dataset} \label{sec:dataset}

% where did dataset come from (cite justin's papers)

% what does raw data look like?
The raw dataset's fields are shown in \cref{table_data}, where (easting,northing) are the latitude/longitude global coordinates, and (x,y) are the coordinates in our global campus map.
Veh id indicates which of the three vehicles corresponds to that data point, or in the pedestrian case, which vehicle sensed that pedestrian, and ped id is a unique id given to each pedestrian seen.

\begin{table}[ht!]
\centering
\begin{tabular}{||c||c c c c c c c||}  
 \hline
 \multirow{1}{*}{Type} &
       \multicolumn{7}{c||}{Fields} \\
 \hline\hline
 Vechicle & time & easting & northing & x & y & veh id & \\ \hline
 Pedestrian & time & easting & northing & x & y & veh id & ped id \\ \hline
\end{tabular}
\caption{Raw data fields}
\label{table_data}
\end{table}

% what are the issues with the raw data
There are some noise-related issues with the raw data, as it was collected on a research vehicle under development.
One issue is that the vehicle's (x,y) position sometimes jumps, because the vehicle's localization system does not use GPS and is imperfect.

In addition to addressing noise, the data also needs to be pre-processed to be useful for our classifier.
Specifically, the pedestrian trajectories must be converted into the vehicle's local frame in order to determine if they cross in front of the vehicle.

% what is strategy to fix issues
The global-to-local transformation relies on knowledge of vehicle orientation (heading angle) and smooth vehicle trajectories, neither of which we have by default.

\begin{algorithm}
	\caption{Algorithm for extracting local trajectories}
	\begin{algorithmic}[1]
		\renewcommand{\algorithmicrequire}{\textbf{Input:}}
		\renewcommand{\algorithmicensure}{\textbf{Output:}}
		\REQUIRE $V_g$, $P_g$: global vehicle, pedestrain trajectories~(\cref{table_data})
		\ENSURE  $P_l$: pedestrian trajectories in local vehicle frame
		\FOR {each vehicle}
			\STATE $I_{pos jump}$ = $\{i \in [1,len(V_g)-1] \mid euclid\_dist(V_g(i)-V_g(i+1))>1.0\}$
			\STATE $I_{time jump}$ = $\{i \in [1,len(V_g)-1] \mid time(V_g(i)-V_g(i+1))>0.5\}$
			\STATE J = $I_{pos jump}$ $\cup$ $I_{time jump}$
			\STATE $T_{valid}$ = $\{[J(i), J(i+1)] \mid time(J(i+1) - J(i)) > 5.0\}$
			% \FOR {each segment in $T_{valid}$}
				% \STATE abc
				% \STATE find smooth veh traj
			% % \ENDFOR
			% \FOR {each $P_{g,i}$ pedestrian id}
			% 	\IF {}
			% 		\STATE vxy = 
			% 	\ENDIF
			% \ENDFOR
		\ENDFOR
		\RETURN pedestrian trajectories 
	\end{algorithmic} 
\end{algorithm}

% what does fixed dataset look like


